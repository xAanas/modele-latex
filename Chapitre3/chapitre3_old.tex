\part{Réalisation}
\setcounter{chapter}{2}
\chapter{Réalisation}
\graphicspath{{Chapitre3/figures/}}
%==============================================================================
\pagestyle{fancy}
\fancyhf{}
\fancyhead[R]{\bfseries\rightmark}
\fancyfoot[R]{\thepage}
\renewcommand{\headrulewidth}{0.5pt}
\renewcommand{\footrulewidth}{0pt}
\renewcommand{\chaptermark}[1]{\markboth{\MakeUppercase{\chaptername~\thechapter. #1 }}{}}
\renewcommand{\sectionmark}[1]{\markright{\thechapter.\thesection~ #1}}

\begin{spacing}{1.2}
%==============================================================================


\section{Outils et langages utilisés}
L'étude technique peut se trouver dans cette partie, comme elle peut \^etre faite en
parall\'ele avec l'\'etude th\'eorique (comme le sugg\'ere le mod\'ele 2TUP).
Dans cette partie, il faut essayer de convaincre le lecteur de vos choix en termes de
technologie. Un \'etat de l'art est souhait\'e ici, avec un comparatif, une synth\'ese et un choix 
d'outils, m\^eme tr\'es brefs.
\section{Pr\'esentation de l'application}
C'est tout \'a fait normal que tout le monde attend cette partie pour coller \'a souhait toutes les images
correspondant aux interfaces diverses de l'application si ch\'ere \'a votre coeur, mais
abstenez vous! Il FAUT mettre des imprime \'ecrans, mais bien choisis, et surtout, c'est
bien de les bien formuler : Choisissez un sc\'enario d'ex\'ecution, par exemple la cr\'eation d'un 
nouveau client, et montrer les diff\'erentes interfaces n\'ecessaires pour le faire, en
expliquant bri\'evement le comportement de l'application. Pas trop d'images, ni trop de
commentaires : concis, encore et toujours.

\'Evitez ici de coller du code : personne n'a envie de voir le contenu de votre classe Java.
Mais  vous  pouvez ins\'erer des snippets (bouts de code) pour montrer certaines
fonctionnalit\'es \cite{YOUSFI2015}\cite{Latex}, si vous en avez vraiment besoin. Si vous voulez montrer une partie de votre code, les \'etapes d'installation ou de configuration vous pourrez les mettre dans l'annexe.
\subsection{Exemple de tableau}

Vous pouvez utiliser une description tabulaire d'une \'eventuelle comparaison entre les travaux existants. Ceci est un exemple de tableau: Tab \ref{tab:exple}.

\begin{table}[ht]
	\centering
	\caption{Tableau comparatif}
	\footnotesize
	\begin{tabularx}{\linewidth}{|>{\bfseries \vspace*{\fill}}X ||>{\centering{}\vspace*{\fill}}X|>{\centering{}\vspace*{\fill}}X|>{\centering{}\vspace*{\fill}}X|>{\vspace*{\fill}}X<{\centering{}}|}	
			\hline 
			& \bfseries Col1 & \bfseries Col2 &\bfseries Col3 &\bfseries Col4\\
			\hline \hline
			Row1		&		&	X	&		&		\\
			Row2		&	X	&		&		&		\\
			Row3		&	X	&	X	&	X	&	X	\\
			Row4		&	X	&		&	X	&	X	\\
			Row5		&	X	&		&	X	&	X	\\
			Row6		&	X	&		&	X	&	X	\\
			Row7		&	X	&		&	X	&		\\
			Row8		&	X	&	X	&	X	&		\\
			\hline
	\end{tabularx}
	\label{tab:exple}
\end{table}

\subsection{Exemple de Code}
Voici un exemple de code Java modifié, avec coloration syntaxique \ref{code:vectorJIF}.

\begin{lstlisting}[rulecolor=\color{white}]
\end{lstlisting}

\begin{lstlisting}[label=code:vectorJIF,caption=Extrait de la classe \textit{Vector.jif},language=java,emph={label L,L}, emphstyle=\bf]
	public class Vector[label L] extends AbstractList[L] { 
		private int{L} length; 
		private Object{L}[]{L} elements;
		public Vector() ... 
		public Object elementAt(int i):{L;i} 
					throws IndexOutOfBoundsException {
			... 
			return elements[i]; 
		}
		public void setElementAt{L}(Object{L} o, int{L} i) ... 
		public int{L} size() { 
			return length; 
		} 
	}
\end{lstlisting}

%==============================================================================
\end{spacing}